\section{Feasibility Analysis}
Feasibility analysis aims to uncover the strengths and weaknesses of 
a project. In its simplest term, the two criteria to judge feasibility 
are cost required and value to be attained. As such, a well-designed 
feasibility analysis should provide a historical background of the 
project, description of the project or service, details of the 
operations and management and legal requirements. Generally, feasibility 
analysis precedes technical development and project implementation. 
There is some feasibility factors by which we can determine that 
project is feasible or not:
\begin{itemize}
\item {\bf{Technical feasibility}}: Technological feasibility is carried 
out to determine whether the project has the capability, in terms of 
software, hardware, personnel to handle and fulfill the user 
requirements. This whole project is based on solving Mathematics equations for which we have used javaScript and to provide output we have used Canvas for providing the output and jspdf.js for pdf genaration and Angular.js for user interface and Node.js as backend. Technical feasibility of this project revolves around the technical boundaries and limitations jspdf.js and Angular.js. But as node.js is secure and structured server side framework, so these languages and technologies are perfect to design the software under this project. Design Reinforcement of Beam(DRB) is technically feasible as it is built up in Open 
Source Environment and thus it can be run on any Open Source platform.
\item {\bf{Economic feasibility}}: Economic analysis is the most 
frequently used method to determine the cost/benefit factor for 
evaluating the effectiveness of a new system. In this analysis we 
determine whether the benefit is gain according to the cost invested 
to develop the project or not. If benefits outweigh costs, only then 
the decision is made to design and implement the system. It is 
important to identify cost and benefit factors, which can be categorized 
as follows:
\begin{enumerate}
\item Development costs.
\item Operating costs.
\end{enumerate}
Design Reinforcement of Beam(DRB) Software is also Economically feasible with 0 Development 
and Operating Charges as it is developed in Angular.js framework, nade.js and jspdf.js  which is FOSS technology and the software is operated on Open 
Source platform.
\item {\bf{Operational feasibility}}: Operational feasibility is a measure 
of how well a project solves the problems, and takes advantage of the 
opportunities identified during scope definition and how it satisfies 
the requirements identified in the requirements analysis phase of system 
development. All the Operations performed in the software are very quick 
and satisfies all the reuirements. This project is also operational feasible as it automates the work of solving the problem of analysising the structures which not only saves time but also saves money as most of the work is done by Employees and M.Tech students is done by this software.
\end{itemize}


